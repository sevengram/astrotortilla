\documentclass[finnish]{article}
\usepackage[finnish]{babel}
%\usepackage{sansseriftitles}
\usepackage{1inchmargins}
%\usepackage[iso]{isodate}
\usepackage{hyperref}
\usepackage{color}
\usepackage{booktabs}
\usepackage{amsmath}
\renewcommand{\familydefault}{\sfdefault}
\renewcommand{\tabcolsep}{8pt}
\usepackage[utf8]{inputenc}

\hypersetup{
    colorlinks=true,
    urlcolor=blue,
}

\newcommand{\surl}[1]{{\small \url{#1}}}

\setlength{\parindent}{0pt}
\setlength{\parskip}{2ex}


\begin{document}


\centerline{\sf \Huge AstroTortilla pikaopas}
\centerline{Päivitetty viimeksi: \today}

Tässä ohjeessa kerrotaan pähkinänkuoressa, miten pääset alkuun AstroTortillan käytössä. Lisätietoja kaikista kohdista löytyy varsinaisesta käyttöohjeesta.

\section{Lataa AstroTortillan asennusohjelma}

Lataa asennusohjelma osoitteesta \surl{http://astrotortilla.sf.net/}. 

Ohjelma asentaa automaattisesti Astrometry.net-ratkojan Cygwin-ympäristöön. Valitse \textbf{Asenna AstroTortilla, Cygwin, astrometry.net ja indeksit.}

Asennusohjelma kysyy minkä tason indeksit haluat asentaa.
Katso ohjeet käyttöohjeen kohdasta 3.3 tai valitse suoraan tasot 4004-4019 (n. 2 Gt).

\section{Käynnistä AstroTortilla}

Valitse oikea kamera ja jalusta AstroTortillan pääikkunassa.

\section{Käänny kohteeseen}

Käänny johonkin syvän taivaan kohteeseen jalustasi GoTo-toiminnolla.

\section{Korjaa suuntaus AstroTortillalla}

Ruksi kaikki kolme valintaa Toiminnot (Actions) -paneelissa ja paina Valota ja ratko (Capture and Solve). Hetken kuluttua nauti prikulleen keskellä kuvaruutuasi paistattelevasta galaksista!

\end{document}
