\documentclass[english]{article}
\usepackage{sansseriftitles}
\usepackage{1inchmargins}
\usepackage[iso]{isodate}
\usepackage{hyperref}
\usepackage{color}


\hypersetup{
    colorlinks=true,
    urlcolor=blue,
}

\newcommand{\surl}[1]{{\small \url{#1}}}

\begin{document}


\centerline{\sf \Huge AstroTortilla User Guide}
\centerline{Latest update: \today}

\tableofcontents

\setlength{\parindent}{0pt}
\setlength{\parskip}{2ex}


\section{Introduction}

AstroTortilla automates repetitive tasks in astrophotography by plate solving
acquired images and making it possible for the necessary software to
communicate with each other. By blind plate solving it is possible to snap a
picture from anywhere on the sky, identify the direction and sync the GoTo
mount to that position.

When doing GoTo-slewing over large distances on the sky the target normally
ends up somewhere else than the center of the camera view, if not completely
out of the frame. AstroTortilla is capable of applying iterative corrections
for the initial slew and getting the target object spot on the center of the
frame, often with one or two slews - no matter how bad the GoTo-calibration.

Plate solving combined with goto control also provides a very fast and
convenient way of polar aligning the mount. Just let AstroTortilla look around
and take a few images and then apply the alignment corrections on your mount�s
alignment knobs as suggested by AstroTortilla. In addition, with the ability of
controlling the mount and camera, AstroTortilla includes a clever photographic
aid for traditional drift alignment with no need for plate solving.

\emph{Mexican tortillas} are commonly prepared with meat to make dishes such as
tacos, burritos, and enchiladas. \emph{AstroTortilla} wraps itself around a
bunch of useful software for a tasty nocturnal snack under the starry night
sky. 

\section{Compatibility with other software and equipment} \subsection{Camera
control}

AstroTortilla is able to control any astronomical camera (CCD, DSLR etc.)
through the remote interface of \textbf{\emph{Nebulosity 2}} by Stark Labs or
\textbf{\emph{Maxim DL}} by Diffraction Limited. Additionally, though not very
useful on the field, AstroTortilla can directly control any compatible camera
with the \hbox{\textbf{\emph{ASCOM interface}}}.

\emph{PHD Guiding} by Stark Labs or any other imaging software can be used via
a \textbf{\emph{screenshot plugin}}. In case your camera can�t be supported in
any of the above ways, AstroTortilla can \textbf{\emph{read your image files
from the hard drive}}. In future versions directory monitoring will be
available to automate the loading of new images.

For users wishing to add support for other camera control software,
encouragement and advice are available.

\subsection{GoTo mount control}

AstroTortilla currently supports any computerized tracking mount with an
\hbox{\textbf{\emph{ASCOM interface}}}.  The field testing of AstroTortilla
functions under the night sky is carried out with EQMOD software controlling
Sky-Watcher HEQ5 and EQ6 mounts.

The ASCOM interface currently limits the usage of AstroTortilla to a Microsoft
Windows platform. Supporting the \emph{INDI} interface standard will hopefully
be possible in the future and enable AstroTortilla to run under Linux/Unix
environments too.

\subsection{Plate solver software}

AstroTortilla currently uses the \textbf{\emph{Astrometry.net}} plate solver
engine. The engine can either be run locally trough the linux-like CygWin
environment for Windows or through the Astrometry.net online plate solve
service. In the future support for using your own local area solver server
might be available.

Another possible engine for plate solving the images is \emph{PinPoint
Astrometric Engine} by DC-3 Dreams, but support for this is not yet
implemented.  Unfortunately AstroTortilla is unable to support the free
\emph{PinPoint LE} since it can not be controlled trough a remote interface.

\section{Plate solving}

Plate solving is the sport of identifying the shapes drawn by the locations of
stars in an astronomical image. Once the stars are identified, this astrometric
solution can be used to work out the coordinates of the center of the frame,
the width an height of the field of view, the rotational angle of the image
with respect to celestial coordinate axes, the angular resolution of one pixel
in the imaging setup and, by knowing the pixel size of the sensor, the exact
focal length of the imaging objective used.

Normally, a stellar catalog, that is, a database of star locations and a rough
estimate of where the telescope was pointed at during exposure are needed to
successfully plate solve an image.

\subsection{Astrometry.net}

The Astrometry.net project is a revolutionary effort of developing means of
working out the astrometric solution of any astronomical image with no initial
knowledge of the imaging parameters \emph{whatsoever}. The project will make
its engine, algorithms and code freely avaible to the public. The key to
blindfolded fast plate solving is reducing the calculation time by prebuilding
huge catalogs of four-star patterns of the whole sky.

At the time of writing Astrometry.net is in beta testing phase. An online user
interface for plate solving your images is at
\surl{http://nova.astrometry.net}.  Astrometry.net can also be run locally in a
Linux, Unix, Mac box or Cygwin for Windows environment.

AstroTortilla is compatible with both of these two ways of using
Astrometry.net. {\color{red}(not true yet, the online API is still unsupported
for now)} For using the online solver with AstroTortilla you need to get
yourself an API key by signing in. You don�t need to sign up in order to sign
in, you can use any OpenID account (Google etc.) instead.

Using the online solver plugin requires your imaging PC to be connected to the
internet and will be slower than local solving. However, it saves you the
trouble from installing Cygwin and keeping a few gigabytes of indices on your
hard drive. The online solver plugin might be especially useful for trying out
AstroTortilla.

\subsubsection{Installing Cygwin and Astrometry.net}

Installing the Linux based Astrometry.net engine locally on a Windows platform,
the Cygwin emulator environment is needed. A single executable Cygwin installer
preconfigured to install Astrometry.net is provided by the AstroTortilla team.

The Cygwin installation will take up some 350 MB of hard disk space and it will
contain many linux-like tools probably not very useful to you.

The indices (databases) needed for plate solving however will use up even more
space. Solving fields shot with camera optics, short refractors etc.
fortunately only require a few small sets of indices, about 1 GB in total.
Indices for solving fields smaller than this grow exponentially in size with
the rest indices taking up about 24 GB in total.

\textbf{1. Getting the installer}

You�ll find the installer at
\surl{http://astrotortilla.sourceforge.net/cygwin-custom/}. Download and run
\\\texttt{cygwin-astrometry-setup.exe}. After running, just repeatedly click
Next until the installer asks you to "Choose A Download Site".

\textbf{2. Choosing downloading sites}

Now you need to select our site for downloading Astrometry.net and another site
for downloading the rest of Cygwin. In the list of the sites, our
\surl{http://astrotortilla.sourceforge.net} site is preselected. Select one
additional site by clicking it holding down the Ctrl key. Any one of them will
do, but a server close to your location might be faster.

Before continuing, \textbf{make sure you have two sites selected!} After that,
keep clicking Next. In any popups that appear, click OK. Finally, after
downloading and installing, click Finish.

\textbf{3. Downloading indices}

Currently obtaining the indices for local plate solving involves accepting the
conditions for using them and sending an email to the Astrometry.net staff as
described in \surl{http://trac.astrometry.net/
browser/trunk/src/GETTING-INDICES}.  Note that for now redistributing the
indices to anyone is forbidden and in the meantime this is the only way to get
them. For "stating clearly your intended use of the indices" in the request
email, it�s fine to explain you like to try out AstroTortilla. The staff will
then shortly contact you with a download link.

\textbf{4. Extracting indices}

After downloading the indices you want, place them in
\texttt{C:\textbackslash cygwin\textbackslash usr\textbackslash
share\textbackslash astrometry\textbackslash data\textbackslash } or wherever
your Cygwin directory is. Now open your Cygwin shell (a shortcut on your
desktop), enter the command \texttt{cd /usr/share/astrometry/data} and then the
command \texttt{tar xjf *.bz2}. Now the FITS-files inside the bz2-files should
be extracted in the data-directory. You can make sure this is done, and then
remove the bz2-files to save space by commanding \texttt{rm *.bz2}. You can
also delete them by hand in Windows. Make sure not to delete the FITS-files!

\textbf{5. Using AstroTortilla to solve images}

If everything went fine, you should now be able to solve images with the
\texttt{solve-field} command in the Cygwin shell. For further info on its
usage, refer to the Astrometry.net documentation.

Now it�s just a matter of choosing \texttt{AstrometryNetSolver} as the solver
engine in AstroTortilla. Be sure to modify the Cygwin path in the settings
table (see below). After this you don�t have to do anything on the command
line.

\section{AstroTortilla usage}

\subsection{Main screen}

The main screen consists of four panels: Telescope, Camera, Solver and Actions.



\textbf{\emph{The Telescope panel}} lets you select a telescope/mount.
Currently, only ASCOMTelescope is available. You can connect to all
ASCOM-compliant telescopes and mounts.

For useful operation, the ASCOM driver of your mount should work as a hub,
i.e., allow several programs to connect to the mount driver simultaneously.
This is the normal operation of, for instance, EQMOD and Astro-Physics
ASCOM-drivers. In case your driver only supports one software at a time,
consider trying out Plain Old Telescope Handset (POTH) at the ASCOM Standards
website, or try some other ASCOM hub.

\textbf{\emph{The Camera panel}} is used to connect to the camera you use.
Depending on your camera control application, select \texttt{MaximDLCamera} or
\texttt{NebulosityCamera}.  

For trying out some other software through a live view on the screen, select
\texttt{CaptureCamera}.  Working settings for PHD Guiding are provided. For
other software, configure the capture area yourself.

For manual GoTo-correction, or other testing purposes, \texttt{FileOpenCamera}
lets you select an image file from the hard drive once you hit Capture and
Solve at the Actions panel.  

\textbf{\emph{The Solver panel}} is where you choose the plate solver software
to use. If you have Cygwin and Astrometry.net installed locally, use
\texttt{AstrometryNetSolver}.

If you plan to use the online-version of Astrometry.net, select
\\\texttt{AstrometryNetWebSolver}. Be sure to enter your personal API-key in
the settings table below.

\textbf{\emph{The Actions panel}} lets you decide what actions to take after
exposing the image and getting a successful plate solution. You can choose only
to Sync the scope (tell it where it�s pointing), Re-slew the scope (turn the
scope where it thought it was pointing before) and even Repeat the cycle until
your within a tolerance of your choice from the desired coordinates.

In the \textbf{\emph{Tools menu}} you can select Goto image or use
AstroTortilla to polar align your mount (see below).

\subsection{Normal workflow}

When you�ve got the telescope, camera and solver selected, slew to your target
with your favourite planetarium software. Now "Target" shows the coordinates of
the object you selected, and "Current" shows the same coordinates since this is
where the mount thinks it pointing at at this stage. This might however be more
or less incorrect.

Select the actions of your choice in the Actions panel (e.g. all three to be
safe), a suitable exposure time in the Camera panel and hit \emph{Capture and
Solve} in the Actions panel. AstroTortilla now uses the camera to expose a
frame and delivers it to the solver engine. Once the engine has solved the
field, AstroTortilla is aware of the true pointing direction of the scope and
indicates the coordinates in the Camera panel.

If the \emph{Sync Scope} option was checked, the true coordinates are indicated
in "Current" at the Telescope panel, too, and the GoTo of your telescope is now
properly calibrated.  

If the \emph{Re-slew to target} option was checked, AstroTortilla applies a new
slew command to the mount, to the coordinates your mount assumed it was
pointing at earlier. Usually your object will now be very close to the center
of your frame.

In case your mount calibration was way off to start with or the object still
isn�t quite centered yet, you can choose to have AstroTortilla repeat the
previous actions automatically until the target is centered. Check the Repeat
until within box and enter your desired tolerance in arc minutes in the numeric
box.

\subsection{Future workflow}

In the future the workflow described above will be greatly simplified when
AstroTortilla will act as an ASCOM-compatible telescope. This means you can
choose AstroTortilla as the telescope driver in your planetarium software. You
can then choose to slew to an object in the planetarium software and
AstroTortilla will automatically apply its plate solving GoTo corrections and
get your object in the center of the frame with one click.

You can also choose not to apply the iterative corrections for slews shorter
than a treshold of your choice, say five degrees. This makes AstroTortilla pass
the slew commands from your planetarium software directly to the ASCOM driver
of your mount to ensure small slews for, e.g., object framing purposes are
executed without delay.

\subsection{Goto image tool}

The Goto image tool solves an image from your hard drive and slews the
telescope to its central coordinates. You can use it to return to the exact
framing of a previous night�s exposing session, or mimic the framing of your
favourite Hubble Space Telescope shot, for example.

To use it, select \emph{Tools $\rightarrow$ Goto image} and select the image to
solve in the appearing dialog.  AstroTortilla then solves it and slew the
telescope to the coordinates of its center. The image has to be in FITS, JPEG,
TIFF or PNM format.

When using the Goto image tool, AstroTortilla ignores any check marks you�ve
entered in the Actions panel and always does iterative centering with the
treshold you entered in the bottom of the panel. The search radius parameter in
your Solver panel is also ignored - the Goto image is always a blind solve.

\subsection{Polar aligning your mount with AstroTortilla}

AstroTortilla provides two ways to help your polar alignment task. You can
access the tools via the Tools menu at the top.

The plate solver based \textbf{Polar Alignment} tool invites you to point the
telescope first in the south and then in the east/west direction. After
pointing, it moves your telescope around a bit and plate solves the directions.
After this, it tells you the amount of your polar alignment error in degrees,
which you then can adjust for.

The \textbf{Drift shot} tool will automate a clever technique for traditional
drift aligning, described below. For this you don�t even need to install a
plate solver, only a camera and a mount supported by AstroTortilla is enough.

In the future, AstroTortilla will employ a fully automatic plate solver
algorithm which takes three images around the sky and leaves you nothing but
the job of turning the alignment knobs. Stay tuned.

\subsection{Polar Alignment tool}

The Polar Alignment tool exposes two pairs of frames with a half a degree
offset in right ascension between the the images of the pair. Their plate
solved coordinates are then used to calculate the polar alignment errors by
detecting the amount of declination error when the mount was only moved along
the RA axis.

A pair of frames shot at the east/west meridian is used to determine the
altitude error of the polar axis, and a pair shot at the southern meridian
(assuming northern hemisphere observation) gets you the azimuthal error.

To use the tool, select \emph{Tools $\rightarrow$ Polar Alignment}. Choose the
correct hemisphere and point your mount roughly at the meridian to measure
azimuth error or to the east or west to measure altitude error (also select
which way you�re pointing in the drop down menu).

Then press the according button to let AstroTortilla shoot and solve the pair
of frames. When done, your polar alignment error in degrees is shown on the
screen. Use this to correct your alignment by turning the alignment knobs on
your mount.

When measuring azimuthal error, be sure not to point within a half a degree or
closer to the meridian to avoid meridian flips! Pointing on the west (right)
side of the southern meridian is the safest bet.

\subsection{Drift shot tool}

Drift alignment is an accurate but tedious way to adjust the polar alignment of
your tracking mount, and it�s easiest to accomplish by using a CCD camera. The
fundamental idea of drift alignment is that polar alignment error will show up
with stars drifting in the declination direction.

When imaging a star close to the \textbf{southern meridian} (assuming northern
hemisphere observation), the error of the polar axis elevation adjustment
contributes is zero and all observed drift is because of \textbf{azimuthal}
misalignment. Similarly when imaging at the \textbf{eastern/western} direction,
only \textbf{elevation} misalignment is responsible for the drift.  These
observations are then used to correct the alignment.

AstroTortilla makes it easier to observe the amount and direction of drift by
employing a technique introduced by Robert Vice in the September 2005 issue of
AstroPhoto Insight magazine. It involves slewing the mount eastward for a
specified amount of time, after which an equally long westward slew is done.

If there�s any declination drift present, the star will trace a horizontal V
pattern. The image is exposed for a few seconds before starting the slews in
order to create a small blob to the other arm of the V to distinguish the
direction of drift.

When the alignment is correct, the V will converge into just a horizontal line.
For more information about the V-drift method, see
\surl{http://www.astrophotoinsight.com/node/568}.  

To use the tool, point your telescope at a reasonably bright star and select
\emph{Tools $\rightarrow$ Drift shot}.  AstroTortilla then starts a 30 second
exposure and moves the mount around during it. You can then examine the
resulting image to see if your alignment causes any declination drift.

\section{Credits}

AstroTortilla is brought to you by

\begin{tabular}{l l} 
Antti Kuntsi (Mickut) &Main software development \\ 
Lauri Kangas (Vostok) &Maxim DL plugin, documentation \\ 
Samuli Vuorinen (naavis) &Polar alignment tool, Astrometry.net online plugin \\
Jussi Kantola (ketarax) &Packaging Astrometry.net for easy installing \\ 
\end{tabular}



\end{document}
